\section{Einleitung}
\label{sec:einleitung}
\subsection{Motivation}
Die Themengebiete Fahrassistenzsysteme und vor allem autonomes Fahren sind heute aktueller denn je. Zum einen, da vieles längst keine Zukunftsmusik mehr darstellt und schon in unserem alltäglichen Leben angekommen ist. Hierzu zählen Systeme zur Überwachung toter Winkel, vom elektrischen Spiegel bis hin zur Rundumsicht im Bird-View-System. Nicht nur Warnsysteme, die den Fahrer auf gefährliche Situationen hinweisen, sondern auch halbautomatische Piloten, die korrigierend eingreifen, sind Realität. Beispiele hierfür sind Lenk- und Parkassistenten, wie auch automatische Notbremssysteme, die nicht nur statische Hindernisse erkennen, sondern sogar die Bewegung kreuzender Autos vorausberechnen können. (Link??) Hierbei scheint sogar oft nicht die Technik, sondern die Rechtslage das begrenzende Element darzustellen.
\newline
Zum anderen, da mit jeder weiteren Funktionalität die Gesamtkomplexität eines möglicherweise autonom fahrenden Fahrzeugs noch deutlicher vor Augen geführt wird. Es werden weitere Probleme erkannt die beachtet und absolut zuverlässig gelöst werden müssen, da im Falle eines Ausfalls oder Fehlverhaltens des Systems Menschenleben auf dem Spiel stehen. Um eine Situation vollständig und korrekt zu erkennen und auszuwerten, ist die Zusammenarbeit unterschiedlichster (mehrerer unterschiedlicher) Sensoren nötig. So werden Autos mit (Stereo-) Kameras, Ultraschall und sogar Radar ausgestattet um auf unterschiedliche Distanz und Richtung Aussagen über die ... *Echtzeit*
\newline
*Überschrift? Das Projektseminar Echtzeitsysteme*
Bereits seit einigen Jahren bietet daher das Fachgebiet Echtzeitsysteme dieses/das gleichnamige Projektseminar an. Schwerpunkte waren dabei stets (halb-) autonomes Fahren und Car2X Kommunikation. Realisiert wurde das mittels eines bzw. mehrerer Modellauto-Chassis mit Motoren und Sensoren, gesteuert von einem 16Bit Mikrocontroller. Um den gesteigerten Anforderungen Rechnung zu tragen, wurden erstmalig dieses Semester die Fahrzeuge mit einer weiteren Platine ausgestattet, die aufgrund des verbauten Prozessors komplexere und rechenaufwändigere Aufgaben ermöglichten, so zum Beispiel Videoverarbeitung zusammen mit der ebenfalls neu angebrachten Kamera.

*Ziele*
*Carolo*

\subsubsection{Undsoweiter}