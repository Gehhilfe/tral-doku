\section{Einleitung}
\label{sec:einleitung}
\subsection{Motivation}
Die Themengebiete Fahrassistenzsysteme und vor allem autonomes Fahren sind heute aktueller denn je. Zum einen, da vieles längst keine Zukunftsmusik mehr darstellt und schon in unserem alltäglichen Leben angekommen ist. Hierzu zählen Systeme zur Überwachung toter Winkel, vom elektrischen Spiegel bis hin zur Rundumsicht im Bird-View-System. Nicht nur Warnsysteme, die den Fahrer auf gefährliche Situationen hinweisen, sondern auch halbautomatische Piloten, die korrigierend eingreifen, sind Realität. Beispiele hierfür sind Lenk- und Parkassistenten, wie auch automatische Notbremssysteme, die nicht nur statische Hindernisse erkennen, sondern sogar die Bewegung kreuzender Autos vorausberechnen können. Hierbei scheint sogar oftmals nicht die Technik, sondern die Rechtslage das begrenzende Element darzustellen.
\newline
Zum anderen, da mit jeder weiteren Funktionalität die Gesamtkomplexität eines möglicherweise autonom fahrenden Fahrzeugs noch deutlicher vor Augen geführt wird. Es werden weitere Probleme erkannt, die beachtet und absolut zuverlässig gelöst werden müssen, da im Falle eines Ausfalls oder Fehlverhaltens des Systems Menschenleben auf dem Spiel stehen. Um eine Situation vollständig zu erkennen und korrekt auszuwerten, ist die Zusammenarbeit mehrerer unterschiedlicher Sensoren nötig. So werden Autos mit (Stereo-) Kameras, Ultraschall und sogar Radar ausgestattet, um auf unterschiedliche Distanz und Richtung Aussagen über die Umgebung machen zu können. Dabei müssen natürlich alle Berechnungen vor Ort und in Echtzeit durchgeführt werden.

%%%%%%%%%%
\subsection{Das Projektseminar Echtzeitsysteme}
Bereits seit einigen Jahren bietet daher das Fachgebiet Echtzeitsysteme das gleichnamige Projektseminar an. Schwerpunkte waren dabei stets (halb-) autonomes Fahren und Car2X Kommunikation. Realisiert wurde dies mittels eines bzw. mehrerer Modellauto-Chassis mit Motoren und Sensoren, gesteuert von einem 16Bit Mikrocontroller. Um den gesteigerten Anforderungen Rechnung zu tragen, wurden erstmalig dieses Semester die Fahrzeuge mit einer weiteren Platine ausgestattet, die aufgrund des verbauten Prozessors komplexere und rechenaufwändigere Aufgaben ermöglichten, so zum Beispiel Videoverarbeitung zusammen mit der ebenfalls neu angebrachten Kamera. So wurde auch das Spektrum der Aufgabenstellung erweitert und erstreckt sich nun von geregeltem Fahren mit Abstandssensoren bis hin zum Anfahren eines Ziels, wobei das Ziel selbstständig mit der Kamera erfasst werden muss. Dabei war die genaue Realisierung der Anforderungen bewusst freigestellt. Auch weitere vertiefende Themen konnten von den Teams selbst ausgewählt werden. Dies ermöglichte ein sehr eigenständiges Arbeiten in der Gruppe, da hier in großem Umfang und für jedes Team individuell, sowohl die Interessen, als auch die Kompetenzen berücksichtigt werden konnten.
\pagebreak \newline
Ein übergeordneter Gedanke war dabei stets die Nachhaltigkeit, damit die Teams in kommenden Semestern von den Ergebnissen profitieren und im Idealfall darauf aufbauen können. Diese Nachhaltigkeit ist sehr wichtig und stellt auch ein Kernthema unserer Ergebnisse dar. Fernziel des Projektseminars ist nämlich das Erarbeiten einer Plattform, die eine Teilnahme am \textit{Carolo Cup}\footnote[1]{https://wiki.ifr.ing.tu-bs.de/carolocup/} oder ähnlichen Veranstaltungen ermöglicht.
