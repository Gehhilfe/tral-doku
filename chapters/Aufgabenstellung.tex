\section{Aufgabenstellung}
\label{sec:aufgabenstellung}

Im Folgenden soll die genaue Aufgabenstellung, sowie unsere daraus abgeleiteten Themen erläutert werden.

%%%%%%
\subsection{Pflichtimplementierung}
Als Pflichtteil des Projektseminars wurde generell das Thema "`Autonomes Fahren mittels Sensorik"', "`Kamera Inbetriebnahme und Erkennung von ArUco-Markern"', sowie als Anwendung von Letzterem das "`Durchfahren von ArUco-Toren"' vorgegeben.  

Aus dem Standardprogramm abgeleitet haben wir aufgrund der örtlichen Rahmenbedingungen festgelegt, dass sich unser Auto autonom in der vorgegebenen Umgebung bewegen können soll. 
Wir haben folgende Annahmen zur Aufgabenstellung getroffen:
\begin{itemize}
	\item Das Auto hat mindestens eine Wand rechts oder links von sich, zu der es seinen Abstand absolut bestimmen kann
	\item Es wird lediglich mit der vorgegebenen Hardware eine Realisierung des autonomen Fahrens durchgeführt, sodass unser Code auch von nachfolgenden Gruppen einfach genutzt werden kann
	\item Das autonome Fahren wird durch einen Zustandsautomaten realisiert
	\item ArUco-Tore bestehen aus zwei Markern, die in beliebigem Abstand (jedoch begrenzt durch den Kamerablickwinkel) voneinander nebeneinander im Flur stehen
	\item Der Winkel von der Verbindungsgerade zwischen den Markern und den Normalen auf den Markern ist nahe $90^\circ$
	\item Es ist kein starkes Gegenlicht vorhanden, da dies den Kontrast der Kamera stark beeinträchtigt
	\item Die AruCo-Tore befinden sich im gleichen Umfeld, das auch für das autonome Fahren definiert ist
\end{itemize}

%%%%%%
\subsubsection{Vertiefungspakete}
Als optionale Vertiefungspakete wurden "`\textit{ROS}-basierte Simulation"', "`Fernsteuerung und Car-2-Car Kommunikation"', sowie "`Inertialsensorik und erweiterte Regelung"' vorgeschlagen.

Von den vorgeschlagenen Möglichkeiten aufgegriffen:
Regelung: für tordurchfahrt
graphisches Programmierung
Zustandsautomat
XBOX-Kontroller (Hobit)


%%%%%%
\subsubsection{GRUNDLAGE!}
Text