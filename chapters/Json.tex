

\section{JSON}
\label{sec:json}
Da bei unserer Lösung die Logik zur Steuerung des Autos, also der
Zustandsautomat, von der konkreten Implementierung getrennt wird, benötigten
wir eine Möglichkeit, diesen Automaten zu erstellen, zu speichern und einzulesen.
Unsere Wahl fiel auf die „\textbf{JavaScript Object Notation}“ (\textit{JSON}), einem
einfachen und kompakten Datenformat, welches alle benötigten Funktionalitäten
mitbringt und für das es bereits einige Parser in verschiedenen
Programmiersprachen gibt. Hauptsächlich war unsere Wahl allerdings durch die
einfache Lesbarkeit des Codes begründet, da wir zu Beginn des Projektes die
Automaten per Hand eintippten.
Eine \textit{JSON}-Datei besteht im Wesentlichen aus Objekten, welche in geschweiften
Klammern gekapselt sind. Diese Objekte bestehen aus beliebig vielen
Eigenschaften, die einen eindeutigen Schlüssel und einen zugeordneten Wert
haben. Arrays von Objekten sind in eckigen Klammern eingeschlossen. Eine
vereinfachte Ansicht eines Automaten könnte also so aussehen:

\begin{figure}[thp]
\begin{lstlisting}[style=json]

{"root": 0,
 "states":
    [{"id": 0,
     "type": "WandFolgen",
     "p": 12,
     "i": 0,
     "d": 30},       
    {"id": 1,
     "type": "FollowWall",
     "p": 12,
     "i": 0,
     "d": 30}],
 "transitions": [ ... ]
}

\end{lstlisting}
\centering
\caption{Aufbau einer einfachen \textit{JSON}-Datei}
\end{figure}

Um diese \textit{JSON}-Datei zu deserialisieren, also konkrete Objekte unserer Zustände
und Transitionen zu erstellen, benötigten wir einen Parser, der dazu in der Lage
ist. Wir verwenden hierzu die \textit{JSON}-Library „\textbf{JSON for modern C++}“ \footnote[1]{https://github.com/nlohmann/json} von Niels
Lohmann.
Die komplette Implementierung dieser Library befindet sich in einer einzigen
Datei, der \texttt{json.hpp}, die in unserer \texttt{FSM.cpp} inkludiert wird.
Für unsere Zwecke verwenden wir den Iterator der Library, um mittels einer
for-Schleife über alle Zustände und Transitionen iterieren zu können und die
\textit{parse}-Funktion, um aus gegebenen Strings ein \textit{JSON}-Objekt erstellen zu können.
Der genaue Vorgang des Einlesens wird im nächsten Abschnitt erläutert.