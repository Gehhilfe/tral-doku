\section{Arbeitsumgebung}
\label{sec:arbeitsumgebung}
Als Arbeitsort stand uns ein studentischer Arbeitsraum des Fachgebiets Echtzeitsysteme zur Verfügung. Hier wurden die Fahrzeuge aufbewahrt und es waren Arbeitsplätze mit Bildschirmen vorhanden, um die Autos oder eigene Laptops anzuschließen. 
\newline
Für Testfahrten wurden jedoch meist die Flure genutzt. Hier konnten die langen Wände zur Orientierung genutzt werden und es stand etwas mehr Platz zur Verfügung, als im Arbeitsraum. Das autonome Fahren gestaltete sich auf den Fluren interessanter, weil hier der recht hohe Wendekreis der Fahrzeuge nicht so ins Gewicht fiel. Außerdem stellten die Flure naturgemäß schon eine realistische Umgebung mit 90° Kurven, Kreuzungen und Hindernissen dar, die natürlich jede Gruppe individuell nutzen konnte. So war es praktisch keine Einschränkung, dass wir teilweise zur genaueren Positionsbestimmung eine Wand in messbarer Nähe voraussetzten. Geeignete Hindernisse und Tore, bestehend aus Pappkartons mit ArUco-Markern, ließen sich natürlich ebenfalls beliebig aufstellen und somit konnten wir immer unterschiedliche Szenarien kreieren. Auch waren von den Vorjahren bereits Querstreifen auf dem Boden angebracht, beispielsweise vor Kurven oder Kreuzungen.
