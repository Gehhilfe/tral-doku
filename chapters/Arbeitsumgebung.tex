\section{Arbeitsumgebung}
\label{sec:arbeitsumgebung}
Als Arbeitsort stand uns ein studentischer Arbeitsraum des Fachgebiets Echtzeitsysteme zur Verfügung. Hier wurden die Fahzeuge aufbewahrt und es waren Arbeitsplätze mit Bildchirmen vorhanden um die Autos oder eigene Laptops anzuschließen. 
Für Testfahrten wurden jedoch meist die Flure genutzt. Sei es weil eine lange Wand zur Orientierung benötigt wurde oder einfach nur etwas mehr Platz zur Verfügung stand als im Arbeitsraum. Auch das autonome Fahren gestaltete sich auf den Fluren interessanter, zumal hier der recht hohe Wendekreis der Fahrzeuge nicht so ins Gewicht fiel. Darüberhinaus stellten die Flure naturgemäß schon eine realistische Umgebung, mit 90° Kurven, Kreuzungen und Hindernissen dar, die natürlich jede Gruppe individuell nutzen konnte. Auch wir nutzten diese Gegebenheit, bei der immer mindestens eine Wand vorhanden ist um Ziele genauer anzufahren.

%%%%%%
\subsection{Noch grundlegender}
Text

%%%%%%
\subsubsection{Grundlage?}
Test

%%%%%%
\subsubsection{GRUNDLAGE!}
Text