\documentclass[12pt,
article,
type=prosem, %sta, diplom, bsc, pp, msc, dr, drfinal, sem, prosem, bsc
colorbacktitle,
instlogo,
accentcolor=tud1c,
%twoside
]{tudthesis}

\usepackage[ngerman]{babel}
%\usepackage[english]{babel}

\usepackage[latin1]{inputenc}
%\usepackage[applemac]{inputenc}

% linebreak for urls in bitex
\usepackage{url}
\urlstyle{rm}

%listings
\usepackage{listings}

\newcommand{\getmydate}{%
\iflanguage{ngerman}{%
  \ifcase\month%
    \or Januar\or Februar\or M\"arz%
    \or April\or Mai\or Juni\or Juli%
    \or August\or September\or Oktober%
    \or November\or Dezember%
  \fi\ \number\year%
}%
\iflanguage{english}{%
  \ifcase\month%
    \or January\or February\or March%
    \or April\or May\or June\or July%
    \or August\or September\or October%
    \or November\or December%
  \fi\ \number\year%
}
}
% changed counter for section wise counting
\usepackage{chngcntr}
\counterwithin{figure}{section} 
\counterwithin{table}{section} 

 
\setinstitutionlogo[height]{logo/ESlogo.png}

\begin{document}
	\counterwithin{lstlisting}{section}
  \thesistitle{Projektseminar Echtzeitsysteme}%
    {Ausarbeitung von Team TRAL}
  \author{Tim Burkert, Robert K�nigstein, Lars Stein, Adrian Weber}
  %do not add your student id!
  %\studentidx{1145456}
	\thesisnumber{ES-B-0060}
  \referee{Clemens v. Loewenich}{Johannes Werner}
	
  \iflanguage{english}{
		\department{	\mbox{Department of Electrical Engineering}\\ and Information Technology (FB18)\\\\Adjunct Member Department of\\ Computer Science (FB20)\\\\Prof. Dr. rer. nat. A. Sch�rr\\\ Merckstra�e 25\\64283 Darmstadt\\\\www.es.tu-darmstadt.de}
		\group{Real-Time Systems Lab}
	}{
		\department{Elektrotechnik und \\Informationstechnik (FB18)\\\\Zweitmitglied Informatik (FB20)\\\\Prof. Dr. rer. nat. A. Sch�rr\\\ Merckstra�e 25\\64283 Darmstadt\\\\www.es.tu-darmstadt.de}
		\group{Fachgebiet Echtzeitsysteme}
	}
  
  \dateofsubmit{\today}
  \makethesistitle
  \affidavit{J. Walker}
	
	%\cleardoublepage
	
	%%% Zusammenfassung vor Inhaltsverzeichnis
	%\begin {abstract}
	%...hallo
	%\end{abstract} 
	
	\cleardoublepage
	

	\pagenumbering{roman}
	\addtocontents{Anhang}{}
	\tableofcontents
	\cleardoublepage
	\listoffigures
	\cleardoublepage
	\listoftables
	\cleardoublepage
	
	\pagenumbering{arabic}
	
	%%% Hier stehen alle von uns eingebundenen Kapitel	
	\section{Einleitung}
\label{sec:einleitung}
\subsection{Motivation}
Die Themengebiete Fahrassistenzsysteme und vor allem autonomes Fahren sind heute aktueller denn je. Zum einen, da vieles längst keine Zukunftsmusik mehr darstellt und schon in unserem alltäglichen Leben angekommen ist. Hierzu zählen Systeme zur Überwachung toter Winkel, vom elektrischen Spiegel bis hin zur Rundumsicht im Bird-View-System. Nicht nur Warnsysteme, die den Fahrer auf gefährliche Situationen hinweisen, sondern auch halbautomatische Piloten, die korrigierend eingreifen, sind Realität. Beispiele hierfür sind Lenk- und Parkassistenten, wie auch automatische Notbremssysteme, die nicht nur statische Hindernisse erkennen, sondern sogar die Bewegung kreuzender Autos vorausberechnen können. (Link??) Hierbei scheint sogar oft nicht die Technik, sondern die Rechtslage das begrenzende Element darzustellen.
\newline
Zum anderen, da mit jeder weiteren Funktionalität die Gesamtkomplexität eines möglicherweise autonom fahrenden Fahrzeugs noch deutlicher vor Augen geführt wird. Es werden weitere Probleme erkannt die beachtet und absolut zuverlässig gelöst werden müssen, da im Falle eines Ausfalls oder Fehlverhaltens des Systems Menschenleben auf dem Spiel stehen. Um eine Situation vollständig und korrekt zu erkennen und auszuwerten, ist die Zusammenarbeit unterschiedlichster (mehrerer unterschiedlicher) Sensoren nötig. So werden Autos mit (Stereo-) Kameras, Ultraschall und sogar Radar ausgestattet um auf unterschiedliche Distanz und Richtung Aussagen über die ... *Echtzeit*
\newline
*Überschrift? Das Projektseminar Echtzeitsysteme*
Bereits seit einigen Jahren bietet daher das Fachgebiet Echtzeitsysteme dieses/das gleichnamige Projektseminar an. Schwerpunkte waren dabei stets (halb-) autonomes Fahren und Car2X Kommunikation. Realisiert wurde das mittels eines bzw. mehrerer Modellauto-Chassis mit Motoren und Sensoren, gesteuert von einem 16Bit Mikrocontroller. Um den gesteigerten Anforderungen Rechnung zu tragen, wurden erstmalig dieses Semester die Fahrzeuge mit einer weiteren Platine ausgestattet, die aufgrund des verbauten Prozessors komplexere und rechenaufwändigere Aufgaben ermöglichten, so zum Beispiel Videoverarbeitung zusammen mit der ebenfalls neu angebrachten Kamera.

*Ziele*
*Carolo*

\subsubsection{Undsoweiter}
	\cleardoublepage
	\section{Grundlegendes zur HW des Auto}
\label{sec:grundlegendesHW}

Hallo?

%%%%%%
\subsection{Noch grundlegender}
Text

%%%%%%
\subsubsection{Grundlage?}
Test

%%%%%%
\subsubsection{GRUNDLAGE!}
Text
	\cleardoublepage
	\section{Grundlegendes zu ROS/Betriebssystem}
\label{sec:grundlegendesROS-OS}

Hallo?

%%%%%%
\subsection{Noch grundlegender}
Text

%%%%%%
\subsubsection{Grundlage?}
Test

%%%%%%
\subsubsection{GRUNDLAGE!}
Text
	\cleardoublepage
	\section{Arbeitsumgebung}
\label{sec:arbeitsumgebung}
Als Arbeitsort stand uns ein studentischer Arbeitsraum des Fachgebiets Echtzeitsysteme zur Verfügung. Hier wurden die Fahrzeuge aufbewahrt und es waren Arbeitsplätze mit Bildschirmen vorhanden um die Autos oder eigene Laptops anzuschließen. 
\newline
Für Testfahrten wurden jedoch meist die Flure genutzt. Hier konnten die langen Wände zur Orientierung genutzt werden und es stand etwas mehr Platz zur Verfügung als im Arbeitsraum. Das autonome Fahren gestaltete sich auf den Fluren interessanter, weil hier der recht hohe Wendekreis der Fahrzeuge nicht so ins Gewicht fiel. Außerdem stellten die Flure naturgemäß schon eine realistische Umgebung mit 90° Kurven, Kreuzungen und Hindernissen dar, die natürlich jede Gruppe individuell nutzen konnte. So war es praktisch keine Einschränkung, dass wir teilweise zur genaueren Positionsbestimmung eine Wand in messbarer Nähe voraussetzten. Geeignete Hindernisse und Tore, bestehend aus Pappkartons mit ArUco-Markern, ließen sich natürlich ebenfalls beliebig aufstellen und somit konnten wir immer unterschiedliche Szenarien kreieren. Auch waren von den Vorjahren bereits Querstreifen auf dem Boden angebracht, beispielsweise vor Kurven oder Kreuzungen.

	\cleardoublepage
	\section{Projektkoordination}
\label{sec:projektkoordination}

Doch nicht nur die technischen Voraussetzungen, sondern auch die Organisation im Team stellte einen wichtigen Faktor für den Erfolg dar.

%%%%%%
\subsection{Verantwortungsbereiche}
Zunächst analysierten wir genau die uns gegebenen technischen Möglichkeiten, als auch die Aufgabenstellung. Daraufhin verteilten wir Verantwortungsbereiche auf die einzelnen Gruppenmitglieder. Einige dieser Bereiche waren:
\begin{itemize}
	\item Code - Verantwortlich für eine lesbare Struktur und Kommentare im Code
	\item Zeitmanagement - Verantwortlich für Zeitplanung und Einhaltung der Fristen
	\item Dokumentation - Verantwortlich für die schriftliche Dokumentation der Team-Absprachen
	\item Vorträge, LateX und einige mehr
\end{itemize}
Es handelte sich dabei ganz bewusst um Verantwortungsbereiche und nicht um Aufgabenteilung. Idee hierbei war, dass die einzelnen Gebiete nicht ausschließlich von dem jeweils Verantwortlichen beachtet oder ausgeführt wurden, sondern die jeweils Zuständigen darauf achteten, dass alle den entsprechenden Rahmen einhielten. 
So konnte sich jeder auf seine Verantwortlichkeiten konzentrieren, ohne fürchten zu müssen zum Beispiel längerfristig die Zeit zu vergessen. Auch die Information über Entscheidungen in der Gruppe konnte so ganz leicht, sowohl im Nachhinein, als auch von abwesenden Teammitgliedern eingesehen werden, ohne dass etwas vergessen oder verpasst wurde.


%%%%%%
\subsection{Zeitplanung und Meilensteine}
Um einen Zeitplan erstellen zu können, notierten wir zunächst alle extern vom Veranstalter vorgegebenen Ziele mit der entsprechenden Deadline als Meilensteine in unserem Zeitplan. Dazu zählen unter anderem die Endergebnisse und auch die beiden Vorträge. Nachdem wir uns mit der Technik vertraut gemacht hatten, konnten wir auch zu den Meilensteinen Zwischenziele mit der benötigten Zeit und der daraus folgenden Deadline abschätzen. Daraus ergab sich ein Zeitplan der die gesamte Bearbeitungszeit über Gültigkeit besaß.

%%%%%%
\subsection{Trello}
Wie bereits erwähnt, legten wir von Anfang an Wert darauf, Absprachen, Entscheidungen aber auch offene Fragen immer schiftlich festzuhalten. Dies sollte natürlich möglichst übersichtlich dargestellt, aber auch stets für alle zugänglich sein. Wir entschieden uns für \textbf{Trello} \footnote[1]{https://trello.com/}. Dies ist ein Online-Organisationsboard, zugänglich über eine Website, womit alle Informationen immer aktuell an einem Ort vorlagen. In Trello können einzelne Karten und Unterkarten erstellt werden um die Themen zu gliedern. Auch Zuordnungen von einzelnen Punkten zu Teilnehmern lassen sich hier realisieren um, zum Beispiel Aufgabenteilung oder Verantwortlichkeiten aufzuzeigen. Unterpunkte kann man abhaken, was sich in einem Fortschrittsbalken ablesen lässt. Ist der gesamte Punkt abgearbeitet lässt auch dieser sich mit einem grünen Haken als erledigt markieren.
*Bild*
	\cleardoublepage
	\section{Aufgabenstellung}
\label{sec:aufgabenstellung}

Hallo?

%%%%%%
\subsection{Noch grundlegender}
Text

%%%%%%
\subsubsection{Grundlage?}
Test

%%%%%%
\subsubsection{GRUNDLAGE!}
Text
	\cleardoublepage
	\section{L\"osung Pflichtimplementierung}
\label{sec:lsgpflicht}

Hallo?

%%%%%%
\subsection{Noch grundlegender}
Text

%%%%%%
\subsubsection{Grundlage?}
Test

%%%%%%
\subsubsection{GRUNDLAGE!}
Text
	\cleardoublepage
	\section{FSM}
\label{sec:fsm}

Wir betrachten nun etwas genauer wir unsere Konzept einer FSM zur 
Kontrollflusssteurung in C++11 umgesetzt haben, dazu gehört neben der Wahl 
der Klassenstruktur auch die Einbindung nützlicher Features von C++11 um die
Speicherverwaltung zu optimieren.


%%%%%%
\subsection{Klassenstruktur}
Zur Modellierung einer FSM haben wir ein Konzept aus einer Kontroller Klasse \texttt{FSM} und zwei abstrakten Basisklassen \texttt{State} und \texttt{Transition} überlegt. Dabei stellen die abstrakten Basisklassen nur eine standardisiertes Interface bereit und erlauben somit einen schnelle Implementation von neuen abgeleiteten Klassen.


Erst die abgeleiteten Klassen implementieren eine genau Funktionalität, wie z.B. einer Wandfolgen oder eine Transition nach einem bestimmten Ereignis. Diese Klassen sind in dem dafür vorgesehen namespace \texttt{TRAL::STATES} und \texttt{TRAL::TRANSITIONS} zu finden.


%%%%%%
\subsubsection{FSM}
Die Klasse \texttt{FSM} implementiert die komplette Kontrollflusssteuerung und kümmert sich ebenso um das Laden einer FSM die zuvor graphisch mit Umlet erstellt wurde. Ebenso hält diese Klasse immer eine aktuelle Referenz zum dem globalen \texttt{MachineState}, in dieser Klasse sind alle Sensorinformationen aufbereitet konsolidiert.

\paragraph{\texttt{FSM::tick}}

Die wichtigste Funktion diese Klasse ist die \texttt{tick} Funktion, diese wird zyklisch von der Rosnode tral-fsm aufgerufen. Dabei wird der Kontrollfluss an den aktuell aktiven State weitergeben. Wenn nun der aktive State einen neuen Ausgabe gesetzt hat und die Kontrolle wieder abgibt werden nun alle an diesen State befindlichen Transition überprüft ob diese Ausgelöst haben, sollte dies der Fall sein wird eine \texttt{transit} vollzogen.

\paragraph{\texttt{FSM::transit}}

Beim Statewechsel wird zuerst dem aktuell noch aktiven State signalisiert das der nun verlassen wird, dabei kann der State
zum Beispiel genutzte Ressourcen wieder freigeben. Darauffolgend wird dem neuen State signalisiert das dieser nun betreten wird und nötige Ressourcen belegen kann.


%%%%%%
\subsubsection{State}
Text


%%%%%%
\subsubsection{Transition}
Text

\subsection{Deserialisieren}

	\cleardoublepage
	\newcommand\JSONnumbervaluestyle{\color{blue}}
\newcommand\JSONstringvaluestyle{\color{red}}

% switch used as state variable
\newif\ifcolonfoundonthisline

\makeatletter

\lstdefinestyle{json}
{
  showstringspaces    = false,
  keywords            = {false,true},
  alsoletter          = 0123456789.,
  morestring          = [s]{"}{"},
  stringstyle         = \ifcolonfoundonthisline\JSONstringvaluestyle\fi,
  MoreSelectCharTable =%
    \lst@DefSaveDef{`:}\colon@json{\processColon@json},
  basicstyle          = \ttfamily,
  keywordstyle        = \ttfamily\bfseries,
}

% flip the switch if a colon is found in Pmode
\newcommand\processColon@json{%
  \colon@json%
  \ifnum\lst@mode=\lst@Pmode%
    \global\colonfoundonthislinetrue%
  \fi
}

\lst@AddToHook{Output}{%
  \ifcolonfoundonthisline%
    \ifnum\lst@mode=\lst@Pmode%
      \def\lst@thestyle{\JSONnumbervaluestyle}%
    \fi
  \fi
  %override by keyword style if a keyword is detected!
  \lsthk@DetectKeywords% 
}

% reset the switch at the end of line
\lst@AddToHook{EOL}%
  {\global\colonfoundonthislinefalse}

\makeatother

\section{JSON}
\label{sec:json}
Da bei unserer Lösung die Logik zur Steuerung des Autos, also der
Zustandsautomat, von der konkreten Implementierung getrennt wird, benötigten
wir eine Möglichkeit, diesen Automaten zu erstellen, zu speichern und einzulesen.
Unsere Wahl fiel auf die „\textbf{JavaScript Object Notation}“ (\textit{JSON}), einem
einfachen und kompakten Datenformat, welches alle benötigten Funktionalitäten
mitbringt und für das es bereits einige Parser in verschiedenen
Programmiersprachen gibt. Hauptsächlich war unsere Wahl allerdings durch die
einfache Lesbarkeit des Codes begründet, da wir zu Beginn des Projektes die
Automaten per Hand eintippten.
Eine JSON-Datei besteht im Wesentlichen aus Objekten, welche in geschweiften
Klammern gekapselt sind. Diese Objekte bestehen aus beliebig vielen
Eigenschaften, die einen eindeutigen Schlüssel und einen zugeordneten Wert
haben. Arrays von Objekten sind in eckigen Klammern eingeschlossen. Eine
vereinfachte Ansicht eines Automaten könnte also so aussehen:

\begin{figure}[thp]
\begin{tabular}{c}
\begin{lstlisting}[style=json]

{"root": 0,
 "states":
    [{"id": 0,
     "type": "WandFolgen",
     "p": 12,
     "i": 0,
     "d": 30},       
    {"id": 1,
     "type": "FollowWall",
     "p": 12,
     "i": 0,
     "d": 30}],
 "transitions": [ ... ]
}

\end{lstlisting}
\end{tabular}
\centering
\caption{Aufbau einer einfachen JSON-Datei}
\end{figure}

Um diese JSON-Datei zu deserialisieren, also konkrete Objekte unserer Zustände
und Transitionen zu erstellen, benötigten wir einen Parser, der dazu in der Lage
ist. Wir verwenden hierzu die JSON-Library „\textbf{JSON for modern C++}“ \footnote[1]{https://github.com/nlohmann/json} von Niels
Lohmann.
Die komplette Implementierung dieser Library befindet sich in einer einzigen
Datei, der „\textbf{json.hpp}“, die in unserer „\textbf{FSM.cpp}“ inkludiert wird.
Für unsere Zwecke verwenden wir den Iterator der Library, um mittels einer
for-Schleife über alle Zustände und Transitionen iterieren zu können und die
„\textit{parse}“-Funktion, um aus gegebenen Strings ein JSON-Objekt erstellen zu können.
Der genaue Vorgang des Einlesens wird im nächsten Abschnitt erläutert.
	\cleardoublepage
	\section{Graphische Programmierung}
\label{sec:graphischeprogrammierung}

Hallo?

%%%%%%
\subsection{Noch grundlegender}
Text

%%%%%%
\subsubsection{Grundlage?}
Test

%%%%%%
\subsubsection{GRUNDLAGE!}
Text
	\cleardoublepage
	\section{Doku}
\label{sec:doku}

Hallo?

%%%%%%
\subsection{Noch grundlegender}
Text

%%%%%%
\subsubsection{Grundlage?}
Test

%%%%%%
\subsubsection{GRUNDLAGE!}
Text
	\cleardoublepage
	\section{Fazit}
\label{sec:fazit}
\subsection{Das Fahrzeug}
Die durch das Projektseminar Echtzeitsysteme gewonnenen Erfahrungen sind sehr vielschichtig. Wie schon in der Einleitung zu vermuten war, galt es, technisch gesehen, Hürden aus den unterschiedlichsten Bereichen zu überwinden. Von Verständnis komplexer Software wie \textit{ROS} und Einsicht in das eingesetzte Linux Echtzeitbetriebssystem zu erlangen, bis zur Änderung der Software auf dem \textit{Fujitsu}-Board. Auf der anderen Seite vom Interpretieren und Verwenden der Sensorwerte bis hin zum Ansteuern der Motoren und ermitteln der Regler-Werte.
\newline
\newline 
Im Laufe der Bearbeitungszeit stellten wir schnell fest, dass vor allem in der Hardware die Schwierigkeiten lagen. Dabei war weniger die Prozessorleistung oder Rechengeschwindigkeit das Problem, diese reichte ohne Probleme auch für das Verarbeiten des Videostroms, als vielmehr die Sensorik, Aktorik und der Aufbau des Fahrzeugs. Selbst mit eigens entwickelter Filterung waren die Ultraschall Werte nicht immer zuverlässig, vor allem, wenn der Winkel zum reflektierenden Objekt ungünstig war, im Besonderen an Ecken und Kanten. Da diese in den Fluren des Instituts jedoch oft ähnlich aufgebaut waren, ließ sich auch hier mit Software ganz gut Abhilfe schaffen. 
\newline
\newline
Deutlich problematischer erwies sich der Hall-Sensor. Da er ungünstig am Fahrzeug positioniert ist, reicht es schon beispielsweise, beim Einschalten den Aufbau versehentlich leicht gegenüber dem Fahrgestell zu verschieben, sodass der Abstand zum im Rad montierten Magneten minimal vergrößert wurde und eine Messung somit unmöglich. Auch traten nicht immer erklärbare Sprünge in den vom Hall-Sensor gelieferten Werten auf, was verständlicherweise ein präzises Fahren sehr erschwerte. 
\newline
\newline
Als recht umständlich erwies sich auch die Lenkung. Nicht der große Wendekreis und das langsame Einlenken, unter anderem verursacht durch den schweren Aufbau, stellten hier die größte Schwierigkeit dar. Sondern vor allem die Nichtlinearität, die zudem in jede Richtung unterschiedlich war. Einen Pfad zu berechnen und das Fahrzeug entsprechend zu steuern stellte sich, wie in Kapitel 7.3 bereits beschrieben, als stellenweise sehr unpräzise bzw. unmöglich heraus. Wir haben daher wieder auf eine Regelung, unterstützt von einer geraden Wand zurückgegriffen, was für unsere Zwecke sehr gut funktionierte, das allgemeine Problem mit der Lenkung jedoch nur umgeht. Die Kamera funktionierte nach korrektem Kalibrieren und Ansprechen mit \textit{ROS} recht zuverlässig. Hier haben wir lediglich den Aufbau etwas modifiziert und den Neigungswinkel der Kamera leicht aufgerichtet, sodass wir vor dem Fahrzeug eine größere Fläche einsehen konnten.

%%%%%%%%%%
\subsection{Projektkoordination und Aufgabenstellung}
Die Projektkoordination war ebenfalls ein wichtiger Aspekt, den wir bedenken mussten. Unser Vorgehen basierte hier auf Erfahrungswerten vorhergehender Gruppenarbeiten. In diesem Projektseminar ergab sich jedoch die Möglichkeit, diese auch über ein ganzes Semester hinweg anzuwenden und zu optimieren, wodurch alle Beteiligten ihren Erfahrungsschatz erweitern konnten. Das Gleiche gilt natürlich auch für das Erstellen und Vortragen von Präsentationen. Dazu hatten wir in dieser Veranstaltung gleich zweimal die Möglichkeit, was natürlich auch sinnvoll ist, da auch die Aufgabenstellung in zwei Blöcke geteilt werden kann. Dabei wurde unser Vorgehen, wie bereits erwähnt, in den zweiwöchentlichen Treffen unterstützt, in denen die wichtigsten Fortschritte und Ziele besprochen wurden. Zu Beginn waren dabei natürlich vornehmlich die Grundlagen und Pflichtimplementierungen oder der Zwischenvortrag Thema. Gegen Ende ging es mehr um die eingeschlagene Richtung und das Thema der vom Team gewählten Vertiefung. Diese musste zwar abgesprochen werden, war jedoch ansonsten komplett freigestellt. Hier war positiv, dass dies nicht nur als Zusatzaufgabe gedacht war. Im Gegenteil stellte die Pflichtimplementierungen eher nur die Basis dar und ermöglichte so eine Einarbeitung in die Thematik. So stand für das selbst gewählte Thema ausreichend Zeit zur Verfügung, um auch neue oder alternative Ideen zu verfolgen.
\newline
Generell ist hier anzumerken, dass die (fast) komplett freie Aufgabenstellung es ermöglichte, dass jedes Team seine eigenen  Interessen im Kontext autonomes Fahren verwirklichen konnte. Dies hat auch den Vorteil, am Ende des Semesters nicht fünf gleiche Projekte zu sehen, sondern vielfältige Lösungen ganz unterschiedlicher Probleme.

%%%%%%%%
\subsection{Unsere Lösung und Ausblick}
Wie bereits in der Einleitung angekündigt und später wieder aufgegriffen, stellt die Nachhaltigkeit und deren praktische Umsetzung das Kernthema unserer Lösung dar. Dies wurde uns schon von Beginn an mit auf den Weg gegeben. Die Wichtigkeit dieses Themas kristallisierte sich jedoch immer mehr heraus. Vor allem bei den bereits angesprochenen Problemen mit der Sensorik und Aktorik darf nicht in jedem Semester von neuem bei null begonnen werden müssen. Dies kostet viel Zeit und bringt einen deutlich geringeren Lerneffekt, als umfangreiche Projekte umzusetzen. Hier ist es sehr wichtig einzelne Aspekte, wie zum Beispiel die Steuerung oder die Parameter für die Regelung (zumindest als Orientierung), korrekt und zuverlässig zur Verfügung zu stellen. Unsere Lösung erlaubt dann, das Verknüpfen solcher einzelnen Einheiten zu einer abstrahierten, sehr modularen Software. Nur so kann auf Dauer eine Entwicklung stattfinden, bei der am Ende auch ein sowohl stabiles als auch nachvollziehbares Ergebnis steht.
\newline
\newline
Gerade im Hinblick auf den Carolo Cup, der als Fernziel des Projektseminars genannt wurde, erscheint es uns sehr wichtig, eine Plattform zu entwickeln, auf der die einzelnen Teams aufbauen  und diese um unterschiedliche Funktionalitäten erweitern können. Dazu bietet unser Konzept der FSM eine sehr einfache, gute und effiziente Lösung, da hier unabhängig verschiedene Funktionalität von verschiedenen Teams entwickelt und einfach verteilt werden kann.
	\cleardoublepage
	%%% Infos zu Latex
	%\input{chapters/infos}
	
  
	%\bibliographystyle{alphadin}
	%\bibliography{literature}
	
	\begin{appendix}
	\section{Erster Anhang}
	\end{appendix}

\end{document}
